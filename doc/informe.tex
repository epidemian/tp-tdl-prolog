\documentclass[xcolor=dvipsnames]{beamer} 
\usecolortheme[named=Brown]{structure} 
\usetheme[height=7mm]{Rochester} 

\setbeamertemplate{items}[default] 

\usepackage[spanish]{babel}
\usepackage[utf8]{inputenc}

\usepackage{color}
\usepackage{listings}
\lstset{ %
language=Prolog,            % choose the language of the code
breaklines=true,            % sets automatic line breaking
frame=single,               % Add a frame to listings.
basicstyle=\footnotesize,   % Font size.
}

% items enclosed in square brackets are optional; explanation below
\title{El lenguaje de programación Prolog}
\subtitle{Presentación para la materia Teoría del Lenguaje}
\author{
Demian Ferrerio \\
Martín Paulucci \\
Axel Straminsky}
\institute[UMBC]{
  Facultad de Ingeniería\\
  Universidad de Buenos Aires \\
}
\date{9 de Mayo, 2011}

\newtheorem{codigo}{Código}

\begin{document}

%--- the titlepage frame -------------------------%
\begin{frame}[plain]
  \titlepage
\end{frame}

\begin{frame}{Visión General}

\begin{itemize}
\item Es el lenguaje más famoso del Paradigma Lógico
\item Un programa Prolog consiste en un conjunto de sentencias, que pueden ser hechos o condiciones.
\item Es conversacional. La interacción con el programa consiste en hacerle preguntas al sistema.
\item Muy usado en Inteligencia Artificial y procesamiento de Lenguajes Naturales.
\end{itemize}

\end{frame}

\begin{frame}{Historia}

\begin{itemize}
\item Su nombre proviene de la abreviación \textit{PROgramming in LOGic}.
\item Fue ideado a principios de los '70 por Colmerauer y Roussel.
\item Nació de un proyecto que no tenía como objetivo la implementación de un lenguaje de programación, sino el procesamiento de lenguajes naturales.
\item Inicialmente interpretado,  hasta que en 1983 se desarrollo un compilador capaz de traducir Prolog a un conjunto de instrucciones de una máquina abstracta denominada Warren Abstract Machine (WAM), y desde entonces Prolog es semi-interpretado.
\end{itemize}

\end{frame}

\begin{frame}{Paradigma Declarativo}
 
\begin{itemize}
 \item El paradigma Lógico, al igual que el Funcional, provienen del paradigma Declarativo.
 \item La programación declarativa permite abstraerse del “cómo” y concentrarse en el “qué” a la hora de escribir programas.
 \item Programar de esta manera tiende a reducir errores(en comparación con la programación imperativa), ya que evita posibles ``efectos colaterales'' que pueda generar el tener que codificar un algoritmo de forma explícita.
 \item Ideal para implementar Computación Paralela
\end{itemize}

\end{frame}

\begin{frame}{Paradigma Lógico}

\begin{itemize}
\item Puramente declarativo, es decir, no tiene estructuras de control.
\item La lógica matemática es la manera más sencilla de expresar formalmente problemas complejos para el intelecto humano.
\item Las resonsabilidades para la ejecución de una tarea están divididas entre el programador, que debe asegurar que el modelo sea lógicamente coherente, y la máquina, que debe resolver el problema de manera eficiente.
\end{itemize}

Los lenguajes lógicos usan sobretodo para las siguientes aplicaciones:

\begin{columns}
  \begin{column}{0.45\textwidth}
  \begin{itemize}
    \item Inteligencia Artificial
    \item Demostración automática de Teoremas
  \end{itemize}
  \end{column}

  \begin{column}{0.45\textwidth}
  \begin{itemize}
    \item Sistemas expertos
    \item Reconocimiento de Lenguaje Natural
  \end{itemize}
  \end{column}
\end{columns}

\end{frame}

\begin{frame}[containsverbatim]{Implementaciones de Prolog}

Algunas implementaciones de Prolog son:

\begin{itemize}
\item \verb SWI-Prolog : soporta Mulithreating.
\item \verb Mercury : Mezcla de Programación Lógica y Funcional.
\item \verb Fprolog : Añade lógica difusa.
\item \verb Prolog++ : Añade Clases y jerarquías de Clases.
\item \verb LogTalk : Añade \textit{POO}.
\item \verb Aprolog : Soporta Polimorfismo y Programación de Alto Nivel.
\end{itemize}

\end{frame}

\begin{frame}[fragile]{Programación en Prolog}

\begin{itemize}
 \item Un programa Prolog puro está compuesto únciamente de un conjunto finito de \textit{Clausulas de Horn}. Hay dos tipos de clausulas: \textit{hechos} y \textit{reglas}.
 \item Un \textit{hecho} define una verdad del programa. Por ejemplo:
\begin{lstlisting}
varon(pedro).
\end{lstlisting}
singifica que Pedro es un varón.

 \item Una \textit{regla} define una relación del tipo: 
\[
(p \wedge q \wedge \cdots \wedge t) \Rightarrow u
\]
 \item En Prolog, se escribe primero el consecuente, y después el o los antecedentes. Por ejemplo:
\begin{lstlisting}
hija (A, B) :- mujer (A), madre (B, A).
\end{lstlisting}
\end{itemize}

\end{frame}


t\begin{frame}[fragile]{Ejemplo: Factorial}
 
Para la codificación de funciones hay que tener en cuenta que:

\begin{enumerate}
 \item No hay flujos de control.
 \item Las funciones son recursivas, el algorimo de control lo hace la máquina subyacente.
 \item Se usan los hechos como condición de corte. 
\end{enumerate}

Veamos un ejemplo:
\lstinputlisting{../src/factorial.pl}



\end{frame}







\end{document}
