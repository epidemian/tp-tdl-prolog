\documentclass[xcolor=dvipsnames]{beamer} 
\usecolortheme[named=Brown]{structure} 
\usetheme[height=7mm]{Rochester} 

\setbeamertemplate{items}[default] 

\usepackage[spanish]{babel}
\usepackage[utf8]{inputenc}


% items enclosed in square brackets are optional; explanation below
\title{El lenguaje de programación Prolog}
\subtitle{Presentación para la materia Teoría del Lenguaje}
\author{
Demian Ferrerio \\
Martín Paulucci \\
Axel Straminsky}
\institute[UMBC]{
  Facultad de Ingeniería\\
  Universidad de Buenos Aires \\
}
\date{9 de Mayo, 2011}


\begin{document}

%--- the titlepage frame -------------------------%
\begin{frame}[plain]
  \titlepage
\end{frame}

\begin{frame}{Visión General}

\begin{itemize}
\item Es el lenguaje más famoso del Paradigma Lógico
\item Un programa Prolog consiste en un conjunto de sentencias, que pueden ser hechos o condiciones.
\item Es conversacional. La interacción con el programa consiste en hacerle preguntas al sistema.
\item Muy usado en Inteligencia Artificial y procesamiento de Lenguajes Naturales.
\end{itemize}

\end{frame}


\begin{frame}{Historia}

\begin{itemize}
\item Su nombre proviene de la abreviación \textit{PROgramming in LOGic}.
\item Fue ideado a principios de los '70 por Colmerauer y Roussel.
\item Nació de un proyecto que no tenía como objetivo la implementación de un lenguaje de programación, sino el procesamiento de lenguajes naturales.
\item Inicialmente interpretado,  hasta que en 1983 se desarrollo un compilador capaz de traducir Prolog a un conjunto de instrucciones de una máquina abstracta denominada Warren Abstract Machine (WAM), y desde entonces Prolog es semi-interpretado.
\end{itemize}

\end{frame}

\end{document}
