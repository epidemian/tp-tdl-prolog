\documentclass[xcolor=dvipsnames]{beamer} 
\usecolortheme[named=Brown]{structure} 
\usetheme[height=7mm]{Rochester} 

\usepackage[spanish]{babel}
\usepackage[utf8]{inputenc}

% items enclosed in square brackets are optional; explanation below
\title{El lenguaje de programación Prolog}
\subtitle{Presentación para la materia Teoría del Lenguaje}
\author{
Demian Ferrerio\\
Martín Paulucci\\
Axel Straminsky}
\institute[UMBC]{
  Facultad de Ingeniería\\
  Universidad de Buenos Aires \\
}
\date{9 de Mayo, 2011}


\begin{document}

%--- the titlepage frame -------------------------%
\begin{frame}[plain]
  \titlepage
\end{frame}

\begin{frame}{A sample slide}

\begin{theorem}[The Poincar\'e inequality]
Suppose $\Omega\in\mathbf{R}^n$ is a bounded domain with smooth
boundary.  Then there exists a $\lambda>0$, depending only on
$\Omega$, such that for any function $f$ in the Sobolev space
$H^1_0(\Omega)$ we have:

\[
  \int_\Omega |\nabla u|^2 \,dx \ge 
  \lambda \int_\Omega |u|^2 \,dx .
\]
\end{theorem}

Here is what \emph{itemized} and \emph{enumerated} lists look like:

\begin{columns}
  \begin{column}{0.45\textwidth}
  \begin{itemize}
    \item itemized item 1
    \item itemized item 2
    \item itemized item 3
  \end{itemize}
  \end{column}

  \begin{column}{0.45\textwidth}
  \begin{enumerate}
    \item enumerated item 1
    \item enumerated item 2
    \item enumerated item 3
  \end{enumerate}
  \end{column}
\end{columns}

\end{frame}

\end{document}
