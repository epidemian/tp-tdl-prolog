% Tamaño de letra.
\documentclass[12pt,titlepage]{report}

%------------------------------ Paquetes ----------------------------------

% Paquetes:

%Para comentarios multilínea.
\usepackage{verbatim}

% Para tener cabecera y pie de página con un estilo personalizado.
\usepackage{fancyhdr}

% Codificación UTF-8
\usepackage[utf8]{inputenc}

% Castellano.
\usepackage[spanish]{babel}

% Tamaño de página y márgenes.
\usepackage[a4paper,headheight=16pt,scale={0.75,0.8},hoffset=0.5cm]{geometry}


% Gráficos:

% Para generar pdf.
\usepackage[pdftex]{graphicx}
\usepackage{pdfpages}

% Para ejemplos de código.
\usepackage{listings}
\lstset{ 
    language=Prolog,            
    breaklines=true,            % Wrappea las lineas automáticamente.
    frame=single,               % Un recuadro en los listings.
    basicstyle=\footnotesize,   % Tamaño de fuente.
}

% Son necesarios?
%\usepackage{float}
%\usepackage{amsmath}
%\usepackage{color}

%------------------------------ ~paquetes ---------------------------------

%------------------------- Inicio del documento ---------------------------

\begin{document}

% ---------------------- Encabezado y pie de página -----------------------

% Encabezado: sección a la derecha.
% Pie de página: número de página a la derecha.

\pagestyle{fancy}
\renewcommand{\sectionmark}[1]{\markboth{}{\thesection\ \ #1}}
\lhead{}
\chead{}
\rhead{\rightmark}
\lfoot{}
\cfoot{}
\rfoot{\thepage}

% ---------------------- ~Encabezado y pie de página ----------------------

% -------------------------- Título y autor(es) ---------------------------

\title{Prolog Hello}
\author{}

% -------------------------- ~Título y autor(es) --------------------------

% ------------------------------- Carátula --------------------------------

\begin{titlepage}

\thispagestyle{empty}

% Logo facultad.
\begin{center}
\includegraphics[scale=0.55]{./fiuba}\\
\textsc{\Large Universidad de Buenos Aires}\\[0.2cm]
\textsc{\Large Facultad de Ingeniería}\\[1.5cm]

% Título central.

\textsc{\large Teoría de Lenguajes (75.31)} \\[0.3cm]
\textsc{\large Trabajo Práctico}

\rule{\linewidth}{0.5mm} \\[0.4cm]
{\huge \bfseries Prolog} \\
\rule{\linewidth}{0.5mm}

% Tabla de integrantes.
\begin{flushleft}
\Large\emph{Integrantes} \\[0.2cm]


% Separación entre columnas.
\begin{tabular}{lll}
Axel Straminsky & XXXXX & axel\_stram@hotmail.com \\
Demian Ferrerio & 88443 & epidemian@gmail.com \\
Martín Paulucci & XXXXX & martin.c.paulucci@gmail.com \\
\end{tabular}
\end{flushleft}

\vfill

% Pie de página de la carátula.
{\large \today}

\end{center}
\end{titlepage}

% ------------------------------- ~Carátula -------------------------------

% -------------------------------- Índice ---------------------------------

% Hago que las páginas se comiencen a contar a partir de aquí.
\setcounter{page}{1}

% Índice.
\tableofcontents
\newpage

% -------------------------------- ~Índice --------------------------------

% ----------------------------- Inicio del tp -----------------------------

\clearpage	
\section{Conclusión}	
Acá va la conclusión... \\
Así se hace una cita \cite{artofprolog}.
	
\clearpage
\begin{thebibliography}{9}
  \bibitem{artofprolog} Leon Sterling \& Ehud Shapiro, \emph{The Art of Prolog: Advanced Programming Techniques}. The MIT Press, 2nd Edition, 1999.
\end{thebibliography}

% ------------------------------ Fin del tp -------------------------------

\end{document}

%---------------------------- Fin del documento ---------------------------




